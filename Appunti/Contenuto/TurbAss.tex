\chapter{Turbine a flusso assiale e misto}

Nel caso delle turbine non si parlerà di flusso assiale puro, è presente anche una significativa componente radiale. Nel caso di una turbina il salto entalpico per stadio è di gran lunga superiore all'analogo elaborabile dal compressore. Entalpia e temperatura decrescono molto rapidamente, l'ipotesi di avere densità costante e l'andamento delle pressioni a gradino tra stadi successivi non è più accettabile a causa proprio della dimensione del salto entalpico. Abbiamo temperature molto elevate, nel compressore è la qualità del design del profilo a dominare mentre nella turbina il limite costruttivo è dato dai materiali della palettatura che lavorano a temperature molto elevate e con deflessioni che vanno dai $50^{\circ} a 180^{\circ}$. 

Il proilo aerodinamico utilizzato in un compressore sarà quindi molto diverso da quello utilizzato in una turbina, quest'ultimo rappresenta più una effettiva variazione di condotto. 

Per andare a vedere come si presenta una turbina assiale facciamo riferimento all'immagine in figura , è rappresentata una turbina a gas ad uso terrestre. Le pale sono attorno ai 45 gradi, immaginando la parte statorica si avrà un grado di reazione di 0.5, probabilmente ad eccezione del primo stadio, a monte potrebbe esserci un IGV. Sono presenti $17$ stadi di compressione e solo $3$ di turbina. 

Vediamo allora quali possono essere le varie configurazioni di turbina. 
\