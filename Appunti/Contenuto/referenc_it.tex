%%%%%%%%%%%%%%%%%%%%%%%% referenc_it.tex %%%%%%%%%%%%%%%%%%%%%%%%%%%%%%
% Esempio di referenze
%
%
% Usare questo file come template per il vostro documento.
%
%%%%%%%%%%%%%%%%%%%%%%%% Springer-Verlag %%%%%%%%%%%%%%%%%%%%%%%%%%

%
% Utenti BibTeX: usare
% \bibliographystyle{}
% \bibliography{}
%
% Non-utenti BibTeX: usare
\begin{thebibliography}{[KLR73]}
%
% ed usare \bibitem per creare referenze.
%
% Usare la sintassi ed il markup seguenti per le vostre referenze.
%
% Monografie
\bibitem[KLR73]{monograph} Kagan, A.M., Linnik, Y.V., Rao, C.R.:
Characterization Problems in Mathematical Statistics. Wiley, New York (1973)

% Contributed Works
\bibitem[Mey89]{contribution} Meyer, P.A.: A short presentation of
stochastic calculus. In: Emery, M. (ed) Stochastic Calculus in
Manifolds. Springer, Berlin Heidelberg New York (1989)

% Journal
\bibitem[MR97]{journal} Miller, B.M., Runggaldier, W.J.: Kalman
filtering for linear systems with coefficients driven by a hidden Markov
jump process. Syst. Control Lett., \textbf{31}, 93--102 (1997)

% Tesi
\bibitem[Ros77]{thesis} Ross, D.W.: Lysosomes and storage diseases. MA
Thesis, Columbia University, New York (1977)

\end{thebibliography}
